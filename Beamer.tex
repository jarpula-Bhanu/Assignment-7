%%%%%%%%%%%%%%%%%%%%%%%%%%%%%%%%%%%%%%%%%%%%%%%%%%%%%%%%%%%%%%%
%
% Welcome to Overleaf --- just edit your LaTeX on the left,
% and we'll compile it for you on the right. If you open the
% 'Share' menu, you can invite other users to edit at the same
% time. See www.overleaf.com/learn for more info. Enjoy!
%
%%%%%%%%%%%%%%%%%%%%%%%%%%%%%%%%%%%%%%%%%%%%%%%%%%%%%%%%%%%%%%%


% Inbuilt themes in beamer
\documentclass{beamer}

% Theme choice:
\usetheme{CambridgeUS}

% Title page details: 
\title{Assignment 7} 
\author{Jarpula Bhanu Prasad - AI21BTECH11015}
\date{\today}
\logo{\large \LaTeX{}}

\usepackage{hyperref}
\usepackage{mathtools}
\usepackage{amssymb}
\usepackage{amsmath}


\begin{document}

% Title page frame
\begin{frame}
    \titlepage 
\end{frame}

% Remove logo from the next slides
\logo{}


% Outline frame
\begin{frame}{Papoulis ch2 problems 2.4}
TABLE OF CONTENTS
    \tableofcontents
\end{frame}


% Lists frame
\section{Question}
\begin{frame}{Problem}
Q)Show that 
\begin{itemize}
    \item  (a) If $Pr(A) = Pr(B) = Pr(AB)$, then $Pr(A\bar{B} + B\bar{A})$ = 0
    \item  (b) If $Pr(A) = Pr(B) = 1$, then $Pr(AB)$ = 1
\end{itemize}
\end{frame}

\section{Solution}
\begin{frame}{Solution}
    a)Given
    \begin{align} \label{1}
        Pr(A) & = Pr(B) = Pr(AB)
    \end{align}
    Now,
    \begin{align} \label{2} 
        A  & = AB + A\bar{B} \\ \label{3}
        Pr(A) & = Pr(AB) + Pr(A\bar{B})
    \end{align}
    form \eqref{1} and \eqref{3} we can say
    \begin{align} \label{4}
         Pr(A\bar{B}) & = 0
    \end{align}
\end{frame}

\begin{frame}
similarly,
    \begin{align} \label{5} 
        B & = AB + B\bar{A} \\ \label{6}
        Pr(B) & = Pr(AB) + Pr(B\bar{A})
    \end{align}
    form \eqref{1} and \eqref{6} we can say
    \begin{align} \label{7}
         Pr(B\bar{A}) & = 0
    \end{align}
\end{frame}

\begin{frame}
    Now from \eqref{4} and \eqref{7} we get
    \begin{align}
        Pr(A\bar{B} + B\bar{A}) &=  Pr(A\bar{B}) + Pr(B\bar{A}) = 0 + 0 \\
        Pr(A\bar{B} + B\bar{A}) &= 0
    \end{align}

    \begin{itemize}
        \item b) Given
    \end{itemize}
    \begin{align} \label{10}
         Pr(A) = Pr(B) = 1
    \end{align}
    Now we know that 
    \begin{align}
       & Pr(A+B)  = Pr(A) + Pr(B) - Pr(AB) \\
    \end{align}
    on rearranging and using \eqref{10}
    \begin{align}\label{13}
        & Pr(A+B) + Pr(AB) = 2 
    \end{align}

\end{frame}

\begin{frame}
    From basics of probability we know that
    \begin{align}\label{14}
        0 \le Pr(A+B) \le 1 \\ \label{15}
        0 \le Pr(AB) \le 1  
    \end{align}

    om comparing \eqref{13} , \eqref{14} and \eqref{15} we can conclude that \\
    $Pr(AB)$ must be equal to 1.
\end{frame}

% Blocks frame
\section{Codes}
\begin{frame}{CODES}
    \begin{block}{Python}
         Download python code from - \href{https://github.com/jarpula-Bhanu/Assignment-7/blob/main/code/verify.py}{Python}
    \end{block}

 \begin{block}{Beamer}
         Download Beamer code from - \href{https://github.com/jarpula-Bhanu/Assignment-7/blob/main/Beamer.tex}{Beamer}
    \end{block}
\end{frame} 

\end{document}